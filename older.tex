% GCIS ESIP talk

\documentclass{beamer}
\usepackage{lmodern}
%\documentclass[notes=only]{beamer}
\usepackage[T1]{fontenc}

\usepackage{fancyvrb}

\setbeamertemplate{navigation symbols}{}
\usetheme{Boadilla}
\usecolortheme{dove}
\beamersetuncovermixins{\opaqueness<1>{25}}{\opaqueness<2->{15}}
\setbeamertemplate{itemize items}[circle]
\setbeamertemplate{enumerate items}[default]
\setbeamertemplate{section in toc}{\inserttocsectionnumber.~\inserttocsection}
\setbeamercolor{subsection in toc}{bg=white,fg=structure}
\setbeamertemplate{subsection in toc}{%
 \hspace{1.3em}{
     \boldmath$\cdot$
  }~\inserttocsubsection\par}

\AtBeginSection[]
{
\begin{frame}<beamer>
  \frametitle{Outline}
  \tableofcontents[currentsection]
\end{frame}
}


\begin{document}
\title[Global Change Information System]{Adaptable Information Models in the\\
Global Change Information System}
\author[Duggan et al]{
    Brian Duggan\inst{1}\inst{2},
    Andrew Buddenberg\inst{3}, \\
    Steve Aulenbach\inst{1}\inst{2},
    Robert Wolfe\inst{1}\inst{4},
    Justin Goldstein\inst{1}\inst{2}
}
\date{December 16, 2014} 
\institute[USGCRP, UCAR]{
\inst{1} US Global Change Research Program \\
\inst{2} University Coporation for Atmospheric Research \\
\inst{3} National Oceanic and Atmospheric Administration \\
\inst{4} National Aeronautics and Space Administration
}

% \begingroup
% \makeatletter
% \setlength{\hoffset}{-.5\beamer@sidebarwidth}
% \makeatother
% \endgroup

\begin{frame}[plain]
\titlepage
\begin{center}
\url{http://data.globalchange.gov}\\
\url{http://github.com/USGCRP/gcis}
\end{center}
\end{frame}

\begin{frame}[plain]
\frametitle{Outline}
\tableofcontents 
\end{frame}

\section{Introduction and Functionality}

\subsection{Mission}
\frame{
    \frametitle{\insertsubsectionhead}
    The US Global Change Research Program (USGCRP) has established
    the Global Change Information System (GCIS) to better coordinate
    and integrate the use of Federal information products on changes
    in the global environment and the implications of those changes for society.
}
\frame{
    \frametitle{\insertsubsectionhead}
    The GCIS provides a RESTful API for retrieving global change information.\\
    The GCIS also provides a triple store.  URLs in the triple store are resolvable
    using the API.  URIs in the triple store are described by the GCIS ontology.
}

\subsection{Support NCA3 report production}

\frame{
    \frametitle{\insertsubsectionhead}
    In May, 2014, the US Global Change Research Program released
    the 2014 National Climate Assessment.\\
    \medskip
    Production of this 829 page report and its web site involved collaboration between
    over 300 authors, numerous editors, graphics producers, scientists, data scientists,
    software developers, and web teams.\\
    \medskip
    The content included 161 findings, 284 figures, 3,395 bibliographic references (journal
    articles, books, reports).\\
    \medskip
     The GCIS facilitated the assembly of the report by providing common {\bf identifiers} for
     resources and concepts, providing a common web interface for entering data, as well as an
     API for accepting data in a variety of formats.
\note{Endnote, Drupal, MS Word, Excel, Google Docs, Graphics Software}
}

\subsection{Serve as the NCA3 website backend}
\frame{
    \frametitle{\insertsubsectionhead}
    A website,  \url{http://nca2014.globalchange.gov}, was released concurrently with the report.
    The site received over 200,000 visits in the first two days after launch and continues to
    receive frequent main stream media attention.\\
    \medskip
    GCIS serves as the backend : the website sends client side requests to \url{http://data.globalchange.gov}
    and receives JSON responses which it uses to populated elements of some pages dynamically.\\
    \medskip
    todo insert picture here
}

\subsection{Provide provenance of resources}
\frame{
    \frametitle{\insertsubsectionhead}
    The GCIS ensures compliance with the Information Quality Act by providing
    traceable identifiers for sources of information.\\
    Given a figure, find the datasets and instruments associated the data behind it.\\
    \medskip
    todo sea level rise graphic
}

\subsection{Be a source of reliable information}
\frame{
    \frametitle{\insertsubsectionhead}
    The inverse of provenance.\\
    \medskip
    Sample questions :\\
    \medskip
    Given a dataset, find reports with figures generated from the dataset.\\
    \medskip
    Show figures associated with data generated by instruments funded by NASA.
    \note{Provenance starts the top and looks down.  We also want to start at the bottom and look up.}
}

\subsection{Connect disparate sources of information}
\frame{
    \frametitle{\insertsubsectionhead}
    Identifiers for resources are curated by various organizations; find ways to link them together.
}

%%%% \frame{
%%%%     Function 1 : support the production of the third National Climate Assessment.
%%%%     
%%%%     One year earlier, in May 2013, the USGCRP had the first public release of the
%%%%     Global Change Information System.\\
%%%%     \medskip
%%%%     The GCIS provides identifiers and a RESTful API for resources used in the construction
%%%%     of the report.\\
%%%%     http://data.globalchange.gov
%%%%     \begin{itemize}
%%%%         \item \href{http://data.globalchange.gov/report/nca3}{/report/nca3}
%%%%         \item \href{http://data.globalchange.gov/report/nca3/finding/extreme-precipitation-increase}{/report/nca3/finding/extreme-precipitation-increase}
%%%%         \item \href{http://data.globalchange.gov/article/10.1080/15287390801997625}{/article/10.1080/15287390801997625}
%%%%     \end{itemize}
%%%% }
%%%% 
%%%% \frame {
%%%%     \frametitle{The Third National Climate Assessment}
%%%%     This 800 page document and its website involved {\bf colloboration} between 300 authors, numerous
%%%%     editors, graphics producers, scientists, data scientists, software developers, and web teams.
%%%% }
%%%% 
%%%% \frame{
%%%%     \frametitle{collaboration}
%%%%     The GCIS facilitated the assembly of the report by providing common {\bf identifiers} for
%%%%     resources and concepts.
%%%% }
%%%% 
%%%% \frame{
%%%%     \frametitle{identifiers}
%%%%     Identifiers are URIs and correspond to explicitly defined concepts.  They can be
%%%%     read or written using a {\bf RESTful API}.
%%%% }
%%%% 
%%%% \frame{
%%%%         \frametitle{RESTful API}
%%%%         The architecture for the GCIS is built around providing :
%%%%         \begin{itemize}
%%%%             \item{a RESTful API}\\
%%%%                 GET a URL for JSON, Turtle or HTML
%%%%             \item{Triple store\\
%%%%                 URIs in the triple store are resolvable URLs in the API.}
%%%%         \end{itemize}
%%%% }
%%%% 
%%%% \frame{
%%%%     \frametitle{architecture}
%%%%     ingest - POST/PUT - relational database \\
%%%%                     -- templates -- turtle - {\bf triple store} \\
%%%%                     -- JSON - API/faceted search \\
%%%% }
%%%% 
%%%% \begin{frame}[fragile]
%%%%     \frametitle{SPARQL}
%%%%      \url{http://bit.ly/gcis-dbpedia}
%%%% \begin{tiny}
%%%% \begin{semiverbatim}
%%%%     PREFIX bibo: <http://purl.org/ontology/bibo/>
%%%%     PREFIX gcis: <http://data.globalchange.gov/gcis.owl#>
%%%%     PREFIX cito: <http://purl.org/spar/cito/>
%%%%     PREFIX dcterms: <http://purl.org/dc/terms/>
%%%%     PREFIX dbprop: <http://dbpedia.org/property/>
%%%%     PREFIX dbpo: <http://dbpedia.org/ontology/>
%%%% 
%%%%     SELECT  DISTINCT ?dbpjournal ?gcisjournal ?issn
%%%%     FROM <http://data.globalchange.gov>
%%%%     WHERE \{
%%%%         SERVICE <http://data.globalchange.gov/sparql> \{
%%%%             ?gcisjournal a bibo:Journal .
%%%%             ?gcisjournal bibo:issn ?issn .
%%%%             ?gcisjournal dcterms:hasPart ?gcisarticle .
%%%%             ?gcisarticle a bibo:Article .
%%%%             ?gcisarticle dcterms:isPartOf ?gcisjournal .
%%%%             ?gcisarticle cito:isCitedBy <http://data.globalchange.gov/report/nca3> .
%%%%          \}
%%%%        SERVICE <http://dbpedia.org/sparql> \{
%%%%         ?dbpjournal dbprop:frequency "Monthly"@en .
%%%%         ?dbpjournal dbpo:issn ?issnd .
%%%%         \}
%%%%       FILTER(?issnd = ?issn)
%%%%     \}
%%%% \end{semiverbatim}
%%%% \end{tiny}
%%%% 
%%%% \end{frame}
%%%% 
%%%% 
%%%% \frame{
%%%%     \frametitle{results}
%%%% 
%%%%     go here
%%%% 
%%%% }
%%%% 
%%%% \frame{
%%%% \frametitle{Resources}
%%%%     GCIDs\\
%%%%     http://data.globalchange.gov
%%%%     \begin{itemize}
%%%%         \item \href{http://data.globalchange.gov/article/10.1080/15287390801997625}{/article/10.1080/15287390801997625}
%%%%         \item \href{http://data.globalchange.gov/report/usfs-pnw-gtr-855}{/report/usfs-pnw-gtr-855}
%%%%         \item \href{http://data.globalchange.gov/reference/007a7014-723e-4ceb-a395-5c986b1bf884}{/reference/007a7014-723e-4ceb-a395-5c986b1bf884}
%%%%         \item \href{http://data.globalchange.gov/report/nca3/figure/global-temperature-and-carbon-dioxide}{/report/nca3/figure/global-temperature-and-carbon-dioxide}
%%%%         \item \href{http://data.globalchange.gov/image/26fc56f4-b4e0-425b-adc8-14c6d961d558}{/image/26fc56f4-b4e0-425b-adc8-14c6d961d558}
%%%%         \item \href{http://data.globalchange.gov/report/nca3/table/decisions-scales}{/report/nca3/table/decisions-scales}
%%%%         \item \href{http://data.globalchange.gov/report/nca3/finding/extreme-precipitation-increase}{/report/nca3/finding/extreme-precipitation-increase}
%%%%         \item \href{http://data.globalchange.gov/organization/nasa}{/organization/nasa}
%%%%         \item \href{http://data.globalchange.gov/person/0000-0001-6667-7047}{/person/0000-0001-6667-7047}
%%%%         \item \href{http://data.globalchange.gov/dataset/nca3-cddv2-r1}{/dataset/nca3-cddv2-r1}
%%%%     \end{itemize}
%%%% }
%%%% 
%%%% 

%% tech
\section{Information Model}
\subsection{Relational}
\frame{
    \frametitle{\insertsubsectionhead}
    A relational model provides one/many to one/many relationships.\\
    One report has many chapters.\\
    One journal has many articles.\\
    Many datasets support many articles.\\
    \medskip
    Critical features : referential integrity, type checking, check constraints,
    cascading updates of primary keys (identifiers), high performance,
    well established scalable tools.
    Postgres also offers some useful features like hstores for looser information.\\
    \medskip
    Closed world assumption for some contexts.  A figure is related to a chapter if and only if it is in the chapter.\\
    \medskip
    The canonical representation of GCIS data is in the Postgres database.\\
    \note{Some advantages are hard to beat, being able to use complex SQL can be really useful}
}

\subsection{Semantic}
\frame{
    \frametitle{\insertsubsectionhead}
    The semantic representation has benefits such as\\
    The ability to describe how two entities are related.\\
    A figure {\bf was influenced by} a report.\\
    An article {\bf was cited by} a report.\\
    A dataset {\bf was derived from} another dataset.\\
    A GCIS ontology defines semantic relationships between entities.\\
    The semantic representation allows GCIS data to be used with external data.\\
    Open world assumption : a figure may be related to a dataset, but this is not in GCIS.\\
}

\subsection{Example}
\begin{frame}[fragile]
    \frametitle{\insertsubsectionhead}
     \url{http://bit.ly/gcis-dbpedia}
\begin{tiny}
\begin{Verbatim}
    PREFIX bibo: <http://purl.org/ontology/bibo/>
    PREFIX gcis: <http://data.globalchange.gov/gcis.owl#>
    PREFIX cito: <http://purl.org/spar/cito/>
    PREFIX dcterms: <http://purl.org/dc/terms/>
    PREFIX dbprop: <http://dbpedia.org/property/>
    PREFIX dbpo: <http://dbpedia.org/ontology/>

    SELECT  DISTINCT ?dbpjournal ?gcisjournal ?issn
    FROM <http://data.globalchange.gov>
    WHERE {
       SERVICE <http://data.globalchange.gov/sparql> {
            ?gcisjournal a bibo:Journal .
            ?gcisjournal bibo:issn ?issn .
            ?gcisjournal dcterms:hasPart ?gcisarticle .
            ?gcisarticle a bibo:Article .
            ?gcisarticle dcterms:isPartOf ?gcisjournal .
            ?gcisarticle cito:isCitedBy <http://data.globalchange.gov/report/nca3> .
       }
       SERVICE <http://dbpedia.org/sparql> 1
        ?dbpjournal dbprop:frequency "Monthly" @en .
        ?dbpjournal dbpo:issn ?issnd .
       }
       FILTER(?issnd = ?issn)
    }
\end{Verbatim}
\end{tiny}

\note{This is a federated SPARQL query that finds articles in GCIS from monthly journals.  Note that
    GCIS does not store the frequency of publication of a journal.  But it does store the ISSN, and dbpedia
    stores the frequency.  So its possible to join GCIS to dbpedia to use this attribute.  Go to
    the URL bit.ly/gcis-dbpedia to actually run this query.}
\end{frame}

%

%%%% \frame{
%%%%     \frametitle{Functionality}
%%%%     \begin{itemize}
%%%%         \item Support NCA3 report production
%%%%         \item Support NCA3 website (client side jQuery)
%%%%         \item Provide minimal landing pages for resources
%%%%         \item Provide a public JSON API \url{http://data.globalchange.gov/api_reference}
%%%%         \item Provide semantic information
%%%%         \item Be interoperable (e.g. use existing identifiers)
%%%%         \item Provide a public SPARQL endpoint \url{http://data.globalchange.gov/sparql}
%%%%         \item JSON, RDF, schema.org, HTML, Turtle, RDF-XML
%%%%     \end{itemize}
%%%% }
%%%% 
%%%% \frame{
%%%%     \frametitle{Testing}
%%%%     \begin{itemize}
%%%%         \item Test driven development (unit tests)
%%%%         \item SPARQL tests
%%%%         \item Continuous Integration Testing (github, travis-ci.org)
%%%%         \item Test driven data acquisition
%%%%         \item Continuous Content Validation\\
%%%%             \url{http://github.com/USGCRP/gcis-qa}
%%%%     \end{itemize}
%%%% }
%%%% 
%%%% \frame{
%%%%     \frametitle{Server Architecture}
%%%%     \begin{itemize}
%%%%         \item RDBMS (PostgreSQL) for storage\\
%%%%             Fine-grained transaction auditing, referential integrity
%%%%         \item HTML templates
%%%%         \item Turtle templates (and other formats)
%%%%         \item Scrape into triple store (Virtuoso)
%%%%         \item Data structures into JSON, YAML
%%%%         \item nginx reverse proxy cache
%%%%     \end{itemize}
%%%% }
%%%% 
%%%% \frame{
%%%%     \frametitle{Clients}
%%%%     \begin{itemize}
%%%%         \item Python (Andrew)\\ \url{http://github.com/USGCRP/gcis-py-client}
%%%%         \item Perl\\ \url{http://github.com/USGCRP/gcis-pl-client}
%%%%         \item Javascript (jQuery)
%%%%         \item php (Drupal)
%%%%     \end{itemize}
%%%% }
%%%% 
%%%% 
%% concepts

\section{System Architecture}

\subsection{Diagram}

\frame{
    \frametitle{\insertsubsectionhead}
    todo
}

\subsection{Schema Changes}

\frame{
    \frametitle{\insertsubsectionhead}
    Changes to the schema propagate to the JSON API.
    JSON key names match the column names, and nested JSON objects correspond to relationships.
    \begin{enumerate}
        \item Write a test for new REST functionality.
        \item Run the tests.  Do they test pass?
        \item Yes?  Done.
        \item No?  Write a schema patch.
        \item Goto step 2.
    \end{enumerate}
    The tests remain part of the test suite, which is run continuously.
    \note{This is the typical methodology for test driven development.  But the implicit
    magic is that running the tests involves patching the database schema on a fresh instance,
    and starting the application.}
}

\subsection{Ontology Changes}
\frame{
    \frametitle{\insertsubsectionhead}
    Change to the triple are handled by turtle templates.
    \begin{enumerate}
        \item Write a test with a SPARQL query that should succeed.
        \item Run the tests.  Do they pass?
        \item Yes?  Done.
        \item No?  Modify the turtle templates.
        \item Go to step 2.
    \end{enumerate}
    The tests remain part of the test suite, which is run continuously.
    \note{This is again the typical test drive development flow.  Here the implicit
        magic is that the test suite constructs a temporary triple story which can
        be queried with SPARQL.}
}
\begin{frame}[fragile]
    \frametitle{\insertsubsectionhead}
    Sample turtle template :
    \begin{Verbatim}[commandchars=\\\{\}]
    {\bf <%= article->uri %>} a gcis:Article;
    {\bf <%= article->uri %>} dcterms:isPartOf
        {\bf <%= article->journal->uri %>};

    \end{Verbatim}
\end{frame}

\subsection{Updating Content}
\frame{
    \frametitle{\insertsubsectionhead}
    todo
    lexicons here
}

%%%% \begin{frame}
%%%%     \begin{center}
%%%%         \Huge Narrative vs structure
%%%%     \end{center}
%%%% \end{frame}
%%%% 
%%%% \begin{frame}
%%%%     \begin{center}
%%%%         \Huge Semantic vs Relational
%%%%     \end{center}
%%%% \end{frame}
%%%% 
%%%% \begin{frame}
%%%%     \begin{center}
%%%%         \Huge Resources
%%%%     \end{center}
%%%% \end{frame}
%%%% 
%%%% \begin{frame}
%%%%     \begin{center}
%%%%         \Huge Identifiers
%%%%     \end{center}
%%%% \end{frame}
%%%% 
%%%% \begin{frame}
%%%%     \begin{center}
%%%%         \Huge Publications, Contributors\\
%%%%         (Entities, Agents, Activities)
%%%%     \end{center}
%%%% \end{frame}
%%%% 
%%%% \begin{frame}
%%%%     \begin{center}
%%%%         \url{http://data.globalchange.gov/resources}
%%%%     \end{center}
%%%% \end{frame}
%%%% 
%%%% \begin{frame}
%%%%     \begin{center}
%%%%         Discussion
%%%%     \end{center}
%%%% \end{frame}
%%%% 
%%%% 

\section{Conclusion, Ongoing Work, Future Plans}

\end{document}

