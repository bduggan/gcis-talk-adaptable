% GCIS ESIP talk

\documentclass{beamer}
%\documentclass[notes=only]{beamer}

\setbeamertemplate{navigation symbols}{}
\usetheme{Warsaw}
\beamersetuncovermixins{\opaqueness<1>{25}}{\opaqueness<2->{15}}

\begin{document}
\title{The Global Change Information System}   
\subtitle{Reports, Data, APIs, Communication and Information}
\author{Brian Duggan\inst{1}\inst{2}, Steve Aulenbach\inst{1}\inst{2}, \\
Robert Wolfe\inst{2}\inst{3}, Justin Goldstein\inst{1}\inst{2}} 
\date{July 10, 2014} 
\institute[shortinst]{\inst{1} UCAR, \inst{2} USGCRP, \inst{3} NASA}

\begin{frame}[plain]
\titlepage
\begin{center}
\url{http://data.globalchange.gov}
\url{http://github.com/USGCRP/gcis}
\end{center}
\end{frame}

\begin{frame}[plain]
\frametitle{Outline}\tableofcontents 
\end{frame}

\section{Assembly}

%% overview

% http://nca2014.globalchange.gov
%   -> explore the report
%   -> our changing climate
%   -> explore our changing climate
%   -> observed change
%   -> Figure 2.2 : CO2 Concentration
%   -> Details/Download
%   -> Image
%   -> datasets used

\subsection{Who}
\frame{
\frametitle{Who}
    \begin{itemize}
        \item Scientists
        \item Science analysts
        \item Editors
        \item Graphic designers
        \item Web developers
        \item Project Managers
    \end{itemize}
}

\subsection{How}
\frame{
\frametitle{How}
    \begin{itemize}
        \item Spreadsheets
        \item Google docs
        \item Email
        \item Endnote
        \item Scientific Software
        \item Graphics Software
        \item Content Management Systems
        \item Wikis
        \item Various miscellaneous desktop and cloud software
    \end{itemize}
\note{
Report writing involves creativity and a wide variety of skills;
there is no one set of tools.
}
}

\subsection{What}
\frame{
\frametitle{What}
    The tools are used to represent and manipulate various resources.
    \begin{itemize}
        \item Journal Articles
        \item Reports
        \item References
        \item Images, Figures, Tables
        \item Findings
        \item Organizations
        \item People
        \item Datasets
    \end{itemize}
}



\subsection{Role of the GCIS}
\frame{
\frametitle{Role of the GCIS}
    \begin{itemize}
        \item Common points of reference
        \item Common vocabulary across teams
        \item Language, terminology, vocabulary, ontology
        \item Uniform Resource Identifiers
        \item URIs are actionable : URLs
        \item Information manipulation via API or web forms
        \item Information extraction via API or browsing
        \item Information modeling with relational or semantic models
        \item Fine grained tracking of all changes.
        \item Convenient useful information entry
        \item Highly scalable information retrieval
    \end{itemize}
}


%% tech

\section{Deployment}
\subsection{Functionality}
\frame{
    \frametitle{Functionality}
    \begin{itemize}
        \item Support nca2014.globalchange.gov (jquery)
        \item GCIS provides JSON backend
        \item Figures have client side calls
        \item Also uses dataset landing pages
        \item API used for data ingestion and retrieval
        \item Support semantic queries
        \item SPARQL endpoint \url{http://data.globalchange.gov/sparql}
        \item JSON, RDF, RDF-A, HTML, Turtle, RDF-XML
    \end{itemize}
}

\subsection{Architecture}
\frame{
    \frametitle{Architecture}
    \begin{itemize}
        \item RDBMS (PostgreSQL) for storage
        \item HTML templates
        \item Turtle templates into other formats
        \item Scrape into triple store
        \item data structures into JSON, YAML
    \end{itemize}
}

\subsection{Server}
\frame{
    \frametitle{Server}
    \begin{itemize}
        \item Perl (mojolicious)
        \item nginx, proxies
        \item postgres
        \item Virtuoso
        \item Caching
    \end{itemize}
}

\subsection{Clients}
\frame{
    \frametitle{Clients}
    \begin{itemize}
        \item Python (Andrew)
        \item Perl
        \item Javascript (jquery)
        \item php (Drupal)
    \end{itemize}
}

\subsection{SPARQL}
\begin{frame}[fragile]
    \frametitle{SPARQL}
     \url{http://bit.ly/1ilgeQz}
\begin{tiny}
\begin{semiverbatim}
PREFIX dbpediaowl: <http://dbpedia.org/ontology/>
PREFIX bibo: <http://purl.org/ontology/bibo/>
PREFIX gcis: <http://data.globalchange.gov/gcis.owl#>
PREFIX cito: <http://purl.org/spar/cito/>

SELECT DISTINCT ?gcisjournal
FROM <http://data.globalchange.gov/sparql>
WHERE {
{
   SERVICE <http://data.globalchange.gov/sparql> {
     ?gcisjournal a bibo:Journal .
     ?gcisjournal bibo:issn ?issn .
     ?gcisarticle gcis:inPublication ?gcisjournal .
     ?gcisarticle cito:isCitedBy <http://data.globalchange.gov/report/nca3> .
   } 
   BIND(STRLANG(?issn, "en") AS ?issn_en)
}  
   SERVICE <http://dbpedia.org/sparql> {
     ?dbpjournal dbpediaowl:frequencyOfPublication "Monthly"@en .
     ?dbpjournal dbpediaowl:issn ?issn_en .
     FILTER(STR(?issn_en) = ?issn)
   } 
}  
\end{semiverbatim}
\end{tiny}

\end{frame}


\subsection{Versioning}
\frame{
    \frametitle{Versioning}
    \begin{itemize}
        \item git
        \item Postgres audit triggers
    \end{itemize}
}

\subsection{Testing}
\frame{
    \frametitle{Testing}
    \begin{itemize}
        \item Test driven development (unit tests)
        \item QA (gcis-qa)
    \end{itemize}
}

%% concepts

\section{Information Model}
\subsection{Concepts}
\begin{frame}
    \begin{center}
        \Huge Narrative vs structure
    \end{center}
\end{frame}

\begin{frame}
    \begin{center}
        \Huge Semantic vs Relational
    \end{center}
\end{frame}

\begin{frame}
    \begin{center}
        \Huge Resources
    \end{center}
\end{frame}

\begin{frame}
    \begin{center}
        \Huge Identifiers
    \end{center}
\end{frame}

\begin{frame}
    \begin{center}
        \Huge Publications, Contributors
    \end{center}
\end{frame}

\subsection{Details}
\begin{frame}
    \begin{center}
        \url{http://data.globalchange.gov/resources}
    \end{center}
\end{frame}

\section{Discussion}
\subsection{Discussion}
\begin{frame}
    \begin{center}
        Thank you
    \end{center}
\end{frame}


\end{document}

